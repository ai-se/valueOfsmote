
%\documentclass[sigconf,review, anonymous]{acmart}
\documentclass[10pt,conference]{IEEEtran}

\usepackage{graphicx}
\usepackage[colorinlistoftodos]{todonotes}
\usepackage{blindtext, graphicx}
\usepackage{hyperref}
\usepackage{caption}
\usepackage{float}
\usepackage{balance}
\usepackage{listings}
\renewcommand\thesection{\arabic{section}}
%\renewcommand\thesubsection{\thesection\arabic{subsection}}
\usepackage{amsmath}
\usepackage{tikz}
\usepackage{comment}
\usepackage{framed}
\usepackage{multirow}
\usepackage{rotating}
\usepackage{bigstrut}
\usepackage{color}
\usepackage{eqparbox}
\usepackage{graphics}
\usepackage{colortbl}
\usepackage{paralist}
\usepackage{algorithm}
\usepackage{algorithmicx}
\usepackage{algpseudocode}
\usepackage{mathptmx} 
\usepackage{picture}
\usepackage[shortlabels]{enumitem}
\usepackage{url}

\newcommand{\bi}{\begin{itemize}[leftmargin=0.4cm]}
	\newcommand{\ei}{\end{itemize}}
\newcommand{\be}{\begin{enumerate}}
	\newcommand{\ee}{\end{enumerate}}

\usepackage{tabularx}
\usepackage{colortbl}
\usepackage{hhline}
\usepackage[export]{adjustbox}
\definecolor{lightgray}{gray}{0.8}
\definecolor{darkgray}{gray}{0.6}
\definecolor{lavenderpink}{rgb}{0.98, 0.68, 0.82}
\definecolor{celadon}{rgb}{0.67, 0.88, 0.69}
\renewcommand{\algorithmicrequire}{\textbf{Input:}}
\renewcommand{\algorithmicensure}{\textbf{Output:}}
%%% graph
\newcommand{\crule}[3][darkgray]{\textcolor{#1}{\rule{#2}{#3}}}

\newcommand{\quart}[4]{\begin{picture}(80,4)%1
	{\color{black}\put(#3,2){\circle*{4}}\put(#1,2){\line(1,0){#2}}}\end{picture}}

\definecolor{Gray}{gray}{0.95}
\definecolor{LightGray}{gray}{0.975}

\definecolor{steel}{rgb}{.11, .11, .7}
\definecolor{Gray}{rgb}{0.88,1,1}
\definecolor{Gray}{gray}{0.85}
\usepackage[framed]{ntheorem}
\usetikzlibrary{shadows}
\theoremclass{Lesson}
\theoremstyle{break}

% inner sep=10pt,
\tikzstyle{thmbox} = [rectangle, rounded corners, draw=black,
fill=Gray!40,  drop shadow={fill=black, opacity=1}]
\newcommand\thmbox[1]{%
	\noindent\begin{tikzpicture}%
	\node [thmbox] (box){%
		\begin{minipage}{.94\textwidth}%
		\vspace{-2mm}#1\vspace{-3mm}%
		\end{minipage}%
	};%
	\end{tikzpicture}}

\let\theoremframecommand\thmbox
\newshadedtheorem{lesson}{Result}

\theoremclass{Lesson1}
\theoremstyle{break}

\let\theoremframecommand\thmbox
\newshadedtheorem{lesson1}{Result}
\newcommand{\tion}[1]{{\S}\ref{sect:#1}}

%\setcopyright{rightsretained}


%\acmConference[FSE'2017]{ACM SIGSOFT SYMPOSIUM ON THE FOUNDATIONS OF SOFTWARE ENGINEERING}{September 2017}{PADERBORN, GERMANY} 
%\acmYear{2017}
%\copyrightyear{2017}

%\acmPrice{15.00}


\begin{document}

\pagestyle{plain}

\title{Software Analytics
without Tuning Considered Harmful:\\A
Case Study with SMOTE}

\author{\IEEEauthorblockN{Amritanshu Agrawal}
\IEEEauthorblockA{Department of Computer Science\\
North Carolina State University\\
Raleigh, NC, USA\\
Email: aagrawa8@ncsu.edu}
\and
\IEEEauthorblockN{Tim Menzies}
\IEEEauthorblockA{Department of Computer Science\\
North Carolina State University\\
Raleigh, NC, USA\\
Email: tim@menzies.us}}
\maketitle


% \author{Amritanshu Agrawal}
% \affiliation{%
%   \institution{North Carolina State University}  \city{Raleigh}
%   \state{North Carolina, USA}}
% \email{aagrawa8@ncsu.edu}

% \author{Tim Menzies}
% \affiliation{%
%   \institution{North Carolina State University}  \city{Raleigh}
%   \state{North Carolina, USA}}
% \email{tim@menzies.us}



\begin{abstract}
When data is highly imbalanced, it is
hard for data miners to find the target
class. Such imbalanced datasets are
common in software engineering,
particularly in the field of defect prediction.


One way to handle imbalance datasets is using the SMOTE procedure (Synthetic Minority Oversampling Technique) which 
throws away random examples of the majority
class while synthesizing
more examples 
of the minority class.
The standard SMOTE,
referred to as SMOTE1,
uses a set of default parameters. Our tuned version
of SMOTE1, referred to as SMOTE2,
uses parameters  found by an optimization technique
called Differential Evolution (DE). 

 In studies with SMOTE1 and SMOTE2, performed on
 9 imbalanced defect datasets, we observe that SMOTE1 may or may not be useful depending on the evaluation goal. We observed mixed results for precision (sometimes better sometimes worse).
On the other hand, SMOTE2 lead to dramatic improvements in  AUC (pf, recall) and good improvements were seen in recall and these were much better than SMOTE1.

We conclude that (a)~SMOTE
is useful but (b)~it is best {\em not} to use the default parameters of standard SMOTE1. More generally, we caution all researchers to tune their analytics tools before
trusting the results they generate.
\end{abstract}

%\keywords{
%Performance Prediction, SBSE, Sampling, Rank-based method}
\begin{IEEEkeywords}
Software repository mining, 
search based software engineering,
defect prediction, classification, 
data analytics for software engineering, SMOTE,  imbalanced data, preprocessing
\end{IEEEkeywords}

\IEEEpeerreviewmaketitle

% \keywords{Software repository mining, 
% search based software engineering,
% defect prediction, classification, 
% data analytics for software engineering, SMOTE,  imbalanced data, preprocessing}

% \maketitle

%\category{H.4}{Software Engineering}{Defect Prediction}
%A category including the fourth, optional field follows...
%\category{D.2.8}{Software Engineering}{Metrics}[complexity measures, performance measures]

%\terms{Theory}


\section{Introduction}


Since time and manpower are finite resources, it
makes sense to assign personnel and/or resources to areas of
a software system with a higher probable quantity of defects~\cite{d2010extensive}. Currently researchers~\cite{song2011general} working in defect prediction focuses on (i) estimating the number of defects remaining in software systems, (ii) discovering defect associations, and (iii) classifying the defect-proneness of software components, typically into two classes defect-prone and not defect-prone. 

This paper deals with the third type of problem for code metrics (especially, the 
CK object-oriented metrics~\cite{chidamber1994metrics}).
Our task is to classify the defect-proneness of software components, typically into two classes, defective and not defective. 

One issue with this kind of research is deciding
which data mining algorithm is best for a particular
dataset. In a comparative study of numerous
learners for defect prediction, Ghotra et al.~\cite{ghotra2015revisiting} recommended six learners: Naive Bayes, Logistic regression, Support Vector Machines, Nearest Neighbor, decision trees and Random forest. One of the limitation of their
analysis was the effects of class imbalance for such models was not explored.
This is an important property
since most software systems have less percentage of defective classes compared to non-defective class. Accordingly, this paper
explores the Ghotra et al. learners using datasets that have an increasingly large imbalances
in the frequency of the defect class. For
example, here we explore datasets where the percent
of defective classes
range down to 2\%. 
Handling class imbalance is important
since it is known to adversely affect
many machine learning algorithms such as decision tress,
neural networks or support vectors machines \cite{japkowicz2002class}. 

Accordingly, we evaluate the performance of learners with imbalanced data using a class balancing technique.
We examine a standard off-the-shelf SMOTE, called SMOTE1, which uses
the default parameters for that algorithm. 
The results were not promising,
so we applied a hyperparameter optimization
method called differential evolution, or DE~\cite{storn1997differential} to tune the parameters of SMOTE1. 
We refer this as tuned version SMOTE2. Hyperparameter optimization chooses a set of hyperparameters for a learning algorithm, usually with the goal of optimizing a measure of the algorithm's performance on an independent dataset.

Using SMOTE1 and SMOTE2, we
explore  the following  questions:  
 
  \textbf{RQ1}: \textbf{Is standard ``off-the-shelf'' SMOTE1 preprocessing method useful for defect prediction?} 

 \begin{lesson}For defect data, SMOTE1 has adverse effect on 
 precision, modest improvements for AUC (pf, recall) and large improvements in recall.
 \end{lesson}

 \textbf{RQ2}: \textbf{Can SMOTE2 achieve better results?} 
 
 \begin{lesson}For defect data, SMOTE2  
 offered    improvements over SMOTE1 for recall
 and dramatic improvements for AUC (pf, recall).
 \end{lesson}
 
 \textbf{RQ3}: \textbf{Does hyperparameter optimization lead to different optimal configurations for different datasets?} 
 
 \begin{lesson}Yes. DE finds different ``best'' parameter settings for SMOTE on different datasets.
 \end{lesson}
  This is an important result
  since it means
  reusing tunings suggested  by  any other  previous study  for a dataset different from the one under study is \underline{{\em not}} recommended. Instead,  it is better to
      use automatic tuning  methods  to find the best tuning parameters for the 
      under study dataset.
      
       Note that
 before we demand that tuning should be a
 standard for all analytics task,
 we must assess the practicality of that
 proposal. This leads to the next question:
 
   \textbf{RQ4}: \textbf{Is tuning 
   impractically
   slow?} 
 
 \begin{lesson}Tuning using DE makes training runtimes about twenty times slower, but given
 the large performance improvements,
 the extra effort is justifiable. \end{lesson}
 
 
 
 \begin{figure*}[t!]
\renewcommand{\baselinestretch}{0.8}\begin{center}
{\scriptsize
\begin{tabular}{c|l|p{4.0in}}
amc & average method complexity & e.g. number of JAVA byte codes\\
\hline
avg, cc & average McCabe & average McCabe's cyclomatic complexity seen
in class\\
\hline
ca & afferent couplings & how many other classes use the specific
class. \\
\hline
cam & cohesion amongst classes & summation of number of different
types of method parameters in every method divided by a multiplication
of number of different method parameter types in whole class and
number of methods. \\
\hline
cbm &coupling between methods & total number of new/redefined methods
to which all the inherited methods are coupled\\
\hline
cbo & coupling between objects & increased when the methods of one
class access services of another.\\
\hline
ce & efferent couplings & how many other classes is used by the
specific class. \\
\hline
dam & data access & ratio of the number of private (protected)
attributes to the total number of attributes\\
\hline
dit & depth of inheritance tree &\\
\hline
ic & inheritance coupling & number of parent classes to which a given
class is coupled (includes counts of methods and variables inherited)
\\
\hline
lcom & lack of cohesion in methods &number of pairs of methods that do
not share a reference to an case variable.\\
\hline
locm3 & another lack of cohesion measure & if $m,a$ are the number of
$methods,attributes$
in a class number and $\mu(a)$ is the number of methods accessing an
attribute,
then
$lcom3=((\frac{1}{a} \sum, j^a \mu(a, j)) - m)/ (1-m)$.
\\
\hline
loc & lines of code &\\
\hline
max, cc & maximum McCabe & maximum McCabe's cyclomatic complexity seen
in class\\
\hline
mfa & functional abstraction & number of methods inherited by a class
plus number of methods accessible by member methods of the
class\\
\hline
moa & aggregation & count of the number of data declarations (class
fields) whose types are user defined classes\\
\hline
noc & number of children &\\
\hline
npm & number of public methods & \\
\hline
rfc & response for a class &number of methods invoked in response to
a message to the object.\\
\hline
wmc & weighted methods per class &\\
\hline
\rowcolor{lightgray}
nDefects & raw defect counts & Numeric: number of defects found in post-release bug-tracking systems.\\
\rowcolor{lightgray}
defects present? & Boolean: if {\em nDefects} $>0$ then {\em true} else {\em false}
\end{tabular}
}
\end{center}
\caption{OO code metrics used for all studies in this paper.
Last line, shown in \textcolor{gray} denote the dependent variable.}
\label{fig:ck}
\vspace{-0.7cm}
\end{figure*}



The rest of this paper is structured as follows:
\tion{review} gives an overview of software defect prediction. \tion{motivation} explains the importance of balancing the minority class.
\tion{smote} introduces SMOTE and discusses how SMOTE has been used in literature. Advantages
of tuning is mentioned in \tion{tune}
and the experimental setup of this paper is discussed in \tion{experiment}
We have answered above research questions in
\tion{results} This is followed by a discussion on the validity of our results 
and a section describing our conclusions and future work.

% \item \textbf{RQ3}: \textbf{Should SMOTE be used ``off-the-shelf'' with their default tunings?}

 

%We created a python package generalised to run any CK metrics based dataset and compare results against 6 learners. Since the classes are imbalanced we used SMOTE~\cite{chawla2002smote} (only on Training Data) which is a synthetic minority over-sampling technique.

%The remainder of the paper is organized as follows. Section \ref{review} gives a brief related work on defect prediction. Section \ref{motivation} talks about why there is a need to balance the data. Since we found astonishing results with smote, section \ref{smote} talks about SMOTE in defect prediction. Experimental setup is provided in section \ref{experiment}. Results are discussed in Section \ref{results}. Threats to validity section is discussed in section \ref{validity}. Final conclusion is being discussed in section \ref{conclusion}. And section \ref{future} talks about our future work.

\section{Related Work}
\subsection{Defect Prediction}
\label{sect:review}

Much prior work has estimated number of defects remaining in software systems~\cite{hall2012systematic} using statistical approaches, capture-recapture 
(CR) models, and detection profile methods (DPM)~\cite{song2011general} or
association rule mining~\cite{song2006software}. A variety of approaches have been proposed to tackle the problem of classifying the defect-proneness of software components. It heavily relied on diverse information, such as code metrics (lines of code, complexity)~\cite{d2010extensive,menzies2007data, nagappan2006mining,shepperd2014researcher,Menzies2010}, process metrics (number of changes, recent activity)~\cite{hassan2009predicting} or previous defects~\cite{kim2007predicting}.

Bird et al.~\cite{bird2009putting} indicate that it is possible to predict which components (for example modules) are likely locations of
defect occurrence using a component's development history,
and dependency structure. Two key properties of software components
in large systems are dependency relationships (which components
depend on or are dependent on by others), and development
history (who made changes to the components and
how many times). Thus, we can link software components
to other components i) in terms of their dependencies, and
also ii) in terms of the developers that they have in common. Prediction models based on the topological properties
of components within them have proven to be quite
accurate~\cite{zimmermann2008predicting}.

By keeping change logs of the most recently or frequently changed files are the most probable source of future defects~\cite{hall2012systematic, catal2009systematic}. These mentioned papers compared various code metrics like CK  metrics  suite,  
McCabes  cyclomatic  complexity, Briands coupling metrics, code metrics, 
dependencies between  binaries. The CK metrics aim at measuring whether a 
piece of code follows OO principles. It contains a check of these OO design 
attributes which are explained in Figure \ref{fig:ck}. CK metrics are a set
of metrics and it has added advantage than other OO and static code attributes metrics~\cite{d2010extensive}. 

There is other added advantage that come with CK metrics is they are  simple  to  compute. CK metrics are endorsed by much of the community since it has been used in past covering research papers about twice (49\%) 
as more traditional source code metrics (27\%) or process metrics (24\%)~\cite{radjenovic2013software}. 

There has been vast amount of studies done to find the best defect prediction performing model. But literature suggests~\cite{lessmann2008benchmarking, ghotra2015revisiting}, that no single prediction technique dominates and making an informed decision is challenging since the techniques are usually applied on datasets with different levels of imbalance, which are preprocessed differently. 
% We highly agree to this given so many variations available in the data band there are so many classification techniques available like Naive Bayes, Rule-Based, Neural Networks, Nearest Neighbour, Support Vector Machines, Decision trees, ensemble methods, to name a few.
The level of imbalance is not the only factor that affects
the performance of the classifiers, but other factors
such as the degree of data overlapping (represented
as duplicates) among the classes, dataset shift (training and test
data follow different distributions), small disjuncts, the lack of density
or small sample size, the
correct management of borderline examples or noisy
data adversely affect the performance of defect predictors~\cite{lopez2014importance,lopez2012analysis}. Many of these problems are related to how to
measure these data characteristics and the quality of
data. For instance, Van et al.~\cite{van2009knowledge}
explored into how the level of noise in data (quality) impact the performance of the classifiers. This makes a necessary argument to study preprocessing filters and how it can affect the performance of classifiers. 

Different data preprocessing has been proved
to improve the performance of defect prediction models by
Menzies et al.~\cite{menzies2007data}. Jiang et al.~\cite{jiang2008can} evaluate the impact of
log transformation and discretization on the performance
of defect prediction models, and report different modeling
techniques ``prefer'' different transformation techniques. For
instance, Naive Bayes achieves better performance on discretized
data, while logistic regression achieves better performance
for both. Peters et al.~\cite{peters2013better} propose different filters; and Li et al.~\cite{li2012sample} propose
to use sampling. Nam et al.~\cite{nam2013transfer} transformed both
training and testing data to the same latent feature space,
and build models on the latent feature space. 
%Too many variables in the datacan result in the ``curse of dimensionality''~\cite{friedman1997bias}.
Feature Selection is a common method that can
reduce features and sampling can balance the diversity of
class instance numbers~\cite{yin2015empirical}, in turn improving the performance of defect prediction. In this paper we only tackle the class imbalance problem.

To tackle the variations available among classifiers, some studies suggested to tune the learners to find the best parameter settings~\cite{tantithamthavorn2016automated, fu2016tuning}.  Both the studies report how performance of a defect predictor is dependent on the parameter settings of the predictors and recommends to use automated ways to tune the predictors.
%According to them every dataset comes with different attributes and with different configurable learners, and we can not use a parameter settings found out by previous studies. Every time a defect prediction learner is used it is recommended to find automatic ways to tune them. 
Recent studies~\cite{fu2016tuning, agrawal2016wrong} endorse DE as a method to perform automated tuning (also known as hyperparameter optimization). This was the motivation to use an automatic method like DE to
tune the settings of SMOTE.

In terms of the paper with most influential work,
our experimental methods to select which learners to pick are informed 
by  Ghotra et al.~\cite{ghotra2015revisiting} on ``Revisiting the impact of classification techniques on the performance of defect prediction models''. To 
compare  the  performance  of  defect prediction  models,  they  used  the  Area  Under  the Receiver Operating Characteristic (ROC) Curve (AUC), which plots  false  positive  rate  against   true  positive rate. We denote this as $AUC\ (pf, recall)$. 
They ran the Scott-Knott test~\cite{jelihovschi2014scottknott} to group classification techniques into statistically distinct ranks. After running these evaluation criteria, they concluded that Naive Bayes, Logistic regression, Support Vector Machines, Nearest Neighbor, decision tree and Random forest performed the best depending on various datasets. Accordingly, 
we use these learners in this paper.

\subsection{Class Imbalance }
\label{sect:motivation}

Class imbalance learning refers to learning from datasets that exhibit significant imbalance among or within classes. Class imbalance  is concerned with the situation in where some classes of data are
highly under-represented compared to other classes~\cite{he2009learning}. By convention,
the under-represented class is called the {\em minority} class,
and correspondingly the class which is over-represented is called the
{\em majority} class. In this paper, we say that class imbalance is worse when ratio of minority class to majority {\em increases}, that is,
{\em class-imbalance of 5:95} is worse than {\em 20:80}.

Misclassifying an example from the minority class is usually more expensive. In software defect prediction, due to the nature of the problem, the defect case is much less likely to happen than the non-defect case.  The failure of finding a defect could degrade software quality greatly (since more bugs
are delivered to the client). Hence,
our learning objective can be described
as \textit{obtaining a classifier that will provide higher performances for the minority class}.

Numerous methods have been proposed to tackle class
imbalance problems at data and algorithm levels. Data-level methods manipulate the training data to select some samples for training. This
include a variety of resampling techniques, random oversampling, random undersampling, and SMOTE~\cite{estabrooks2004multiple}. Algorithm-level methods address class imbalance by modifying their training mechanism directly with the 
goal of better accuracy on the minority class, including cost-sensitive learning algorithms ~\cite{he2009learning}.
Algorithm-level methods require specific treatments for different
kinds of learning algorithms, which hinders their use
in many applications, because we do not know in advance
which algorithm would be the best choice in most cases. In addition to the aforementioned data-level and algorithm-level solutions, ensemble learning has become another promising category of approaches to handle imbalanced data by combining multiple classifiers, such as SMOTEBoost~\cite{chawla2003smoteboost}, and
AdaBoost.NC~\cite{wang2010negative}. To 
the best of our knowledge, none of these methods have  thoroughly investigated the class imbalance problem.

Hall et al.~\cite{hall2012systematic} found that models based on C4.5 underperform if they have imbalanced data while Naive Bayes and Logistic regression perform relatively better. 
Their general recommendation is to not use
imbalanced data.  

Yu et al.~\cite{yuperformance} validated the Hall et al. results and concluded that the
performance of C4.5 is unstable on imbalanced datasets. They studied the stability issues due to  class imbalance and found out that Random Forest and Naive Bayes are the most stable among the learners explored. They generated synthetic datasets from the original one to build class imbalance datasets. This could be affected by random sampling and heance may not be the ideal way to build imbalanced datasets. They also used default set of parameters but wanted to study the effects of tuning classifiers as well as mentioned to use techniques like SMOTE for class imbalance in their future work.

Wang et al.~\cite{wang2013using} studied various undersampling and oversampling technique and compared the results with Naive Bayes and Random Forest. Other interesting finding from Wang et al. was to use AdaBoost.NC, which has better performance than the rest while others like Gray et al.~\cite{gray2009using} is planning to use SMOTE in future studies. Yan et al.~\cite{yan2010software} performed fuzzy logic and rules to overcome the imbalance problem only to work with Support Vector Machines. 

Pelayo et al.~\cite{pelayo2007applying} studied the effects of percentage of oversampling and undersampling done. They found out that different percentage of each helps improve the accuracies of decision tree learner for defect prediction using CK metrics. Menzies et al.~\cite{menzies2008implications} undersampled the non-defect class to balance training
data, and report how little information was required to learn a defect predictor. They found that throwing away data does not degrade the performance of Naive Bayes and C4.5 decision trees, and instead improves the performance of C4.5. Some other papers also showed the usefulness of resampling based on different learners~\cite{pelayo2007applying, pelayo2012evaluating, riquelme2008finding}.

\subsection{On the Value of SMOTE for SE}
\label{sect:smote}

This section lists some of the results achieved by SMOTE for SE datasets. Some authors~\cite{van2007experimental, tan2015online} found SMOTE to be advantageous, others~\cite{pelayo2007applying} do not. This 
difference in opinion was one of the main motivations
for this paper.

SMOTE works by creating a new minority-class sample at a random point along the line
segments joining any/all of the $k$ minority class nearest neighbors. Depending upon the
amount ($m$, number of samples) of over-sampling required, neighbors from the $k$ nearest neighbors are randomly
chosen. To find nearest neighbors different distance metric is used. Most common distance metric Minkowski~\cite{deza2009encyclopedia} which has power ($r$) in its formula.

Now coming to the use cases of SMOTE in defect prediction, Pears et al.~\cite{pears2014synthetic} used SMOTE to study software build outcomes. They observed
that classification accuracy steadily improves after creating approximately 900 instances of builds that have been fed to the classifier. Tan et al.~\cite{tan2015online} investigated online defect prediction for imbalance data. They studied resampling techniques to remove imbalance improving precision by 6.4 - 34.8\%.

Pelayo~\cite{pelayo2007applying} found out that by using SMOTE, there was no improvement in defect prediction but other resampling strategies like trial-and-error, showed promising results by arriving at the highest geometric mean accuracies. Kamei et al.~\cite{kamei2007effects} evaluated the effects of SMOTE applied to only four fault-proneness models
(linear discriminant analysis, logistic regression
analysis, neural network and classification tree) by
using two module sets of industry legacy software. They reported that SMOTE improved the prediction performance of the linear and logistic models, while neural network and classification tree models did not
benefit from it. Van et al.~\cite{van2007experimental} identified that classifier performance is improved with SMOTE, but individual learners respond differently on sampling. We will be studying many more models and will show that SMOTE does help in all prediction models depending on tuning goal.

\subsection{Importance of Tuning}
\label{sect:tune}

The previous section documented a strange difference of opinion
about the value of SMOTE. One explanation for these differing results 
is that SMOTE runs with different control parameters.
The impact of tuning a learner's control parameters is well understood in the theoretical machine learning literature~\cite{bergstra2012random}.  When we tune a
data miner for its performance, what we are really doing is changing how a learner applies its
heuristics. This means tuned data miners use different heuristics, which means
they ignore different possible models, which means they return different models;
i.e. \textit{how} we learn changes \textit{what} we learn.

Yet issues relating to
tuning are poorly addressed in the software analytics literature. Fu et al.~\cite{fu2016tuning} surveyed a few of recent SE papers in the area
of software defect prediction from static code attributes and found that SE
  authors, with the exception of~\cite{tantithamthavorn2016icse} do not take steps to explore tunings. For example, Elish et
  al.~\cite{elish2008predicting} compared support vector machines to other data
  miners for the purposes of defect prediction. They tested different
  ``off-the-shelf'' data miners on the same dataset, without adjusting the
  parameters of each individual learner. Similar comparisons of data miners in SE,
with no or minimal pre-tuning study, can be found in the work on Lessmann et al.~\cite{4527256}
and, most recently, in Yang et al~\cite{Yang:2016}.  

We choose to use DE after a literature search on search-based SE methods. DE is a method that optimizes a problem by iteratively trying to improve a candidate solution with regard to a given measure of quality.
In the ast researchers have used many optimizers like: simulated
annealing~\cite{feather2002converging, menzies2007business}; various genetic
algorithms~\cite{goldberg1979complexity} augmented by techniques such as
DE (differential evolution~\cite{storn1997differential}), tabu search and scatter
search~\cite{glover1986general, beausoleil2006moss, molina2007sspmo,nebro2008abyss}; particle swarm optimization~\cite{pan2008particle}; numerous
decomposition approaches that use heuristics to decompose the total space into
small problems, then apply a response surface methods~\cite{krall2015gale, zuluaga2013active}.
Of these, we use DE for two reasons. Firstly, it has been proven useful in prior SE tuning
studies~\cite{fu2016tuning, agrawal2016wrong}. Secondly, our reading of the current literature is
that there are many advocates for differential evolution.

Fu et al.~\cite{fu2016tuning} showed that using DE to tune the parameters of software defect prediction learners results in large improvements, and the tuning was really simple. Similarly Agrawal et al.~\cite{agrawal2016wrong} tuned the parameters of LDA (Latent Dirichlet Allocation~\cite{blei2003latent}) and reports how tuning can provide better model stability. This shows how advantageous it is to tune the parameters of a data miner.

\section{Experimental Setup}
\label{sect:experiment}

All the data, source code and results can be found online%\footnote{Blinded for peer review}.
\footnote{https://github.com/ai-se/Smote\_tune}.
We employ 3 techniques which we have named as, (a) Baseline learner results without SMOTE (No-SMOTE), (b) Standard SMOTE with default parameters (SMOTE1) and (c) SMOTE1 tuned using DE (SMOTE2).  Our implementation currently uses 5 nearest neighbors for SMOTE1 method and for SMOTE2, DE finds the best $k$. Synthetic samples
are generated in the following way:
\bi
\item
Take the difference between the feature vector (sample)
under consideration and its nearest neighbor.
\item
Multiply this difference by a random number
between 0 and 1, and add it to the feature vector under consideration.
\item
This causes the
selection of a random point along the line segment between two specific features
\ei

To reduce the variation, each of the above techniques is performed with a 5-fold stratified cross validation~\cite{refaeilzadeh2009cross} which means:
\bi
\item We randomized the order of the training set five times;
\item Each time, we divided the data into five bins;
\item For each bin (test), we trained on 4 bins (rest) and then tested
on the test bin as follows:
\bi
\item
On the training set, SMOTE's super-sampling selects instances from the minority class and finds ``$k$'' nearest neighbors for each instance and then creates new instances using the selected instances and their neighbors until we have ``$m$'' numbers of minority class samples. 
In our data, 
 the minority class is considered the Defective class and all the datasets used here have less than 50\% minority samples.
\item
On the training set, SMOTE's sub-sampling  eliminates instances from the majority class (selected at random)
until we have ``$m$'' remaining samples.
 
\item On the test set, we do nothing (i.e. no SMOTE;
i.e. we only SMOTE the training samples
(leaving the  testing data in its natural form).
\ei
\ei

\subsection{\textbf{Data}}
 We used the datasets available in PROMISE repository\footnote{http://openscience.us/repo/defect/ck/}~\cite{promiserepo}. In total, 9 imbalanced datasets are used, which were collected by Jureczko et al.~\cite{jureczko2010towards}. Statistics of these datasets can be found in Table~\ref{tb:dataset}. Datasets are sorted with low percentage of defective class to high defective class. These datasets have already been used in various defect prediction case studies by various researchers~\cite{he2012investigation,peters2013better,peters2013balancing,turhan2013empirical} making these datasets more important.
 
 \begin{table}[!htbp]
\begin{center}
\begin{tabular}{|c|c|c|c|}
\hline 
\textbf{Version} & \textbf{Dataset Name} &  \textbf{Defect \%} & \textbf{Non-Defect \%}\\[0.5ex]
\hline
4.3 & jEdit & 2 & 98 \\
\hline
1.0 & Apache Camel & 4 & 96 \\ 
\hline
6.0.3 & Apache Tomcat & 9 & 91 \\
%\hline
%6 & Proprietary Projects & 10 & 90 \\
\hline
2.0 & Apache Ivy & 11 & 89 \\ 
\hline
1.0 & Arcilook & 11.5 & 88.5\\
\hline
1.0 & Redaktor & 15 & 85 \\
\hline
1.7 & Apache Ant & 22 & 78 \\ 
\hline
1.2 & Apache Synapse & 33.5 & 66.5 \\
\hline
1.6.1 & Apache Velocity & 34 & 66 \\
\hline
%3.0 & Apache Poi & 63 & 37 \\
%\hline
%1.4.4 & Xerces & 74 & 26 \\
%\hline
%1.2 & Apache log4j & 92 & 8 \\
%\hline
%2.7.0 & Apache Xalan & 99 & 1 \\
%\hline
\end{tabular}
\end{center}
\caption{Dataset Statistics}
\label{tb:dataset}
\vspace{-0.7cm}
\end{table}

\subsection{\textbf{Preprocessing}}
 We ignored  string columns in the data and assumed that the last column in the dataset is always the target class. Originally, the target class contains number of defects, which we converted to binary, i.e if target class has defect then it represents 1 otherwise it denotes 0. The code assumes user has preprocessed the data before passing it to the learners. 

\subsection{\textbf{Classifiers}}
\label{sect:classes}

We used six classifiers which are mentioned in the baseline paper~\cite{ghotra2015revisiting}
using the default
parameters  suggested by Ghotra et al.

\bi
 \item \textbf{Support Vector Machines (SVM - Linear Kernel)} are supervised learning models with associated learning algorithms that analyze data used for classification and regression analysis. Given a set of training examples, each marked as belonging to one or the other of two categories, an SVM training algorithm builds a model that assigns new examples to one category or the other, making it a non-probabilistic binary linear classifier.
 \item \textbf{Linear Regression (LR)} is an approach for modeling the relationship between a scalar dependent variable $y$ and one or more explanatory variables (or independent variables) denoted $x$.
 \item \textbf{Naive Bayes (NB)} classifiers are a family of simple probabilistic classifiers based on applying Bayes theorem with strong (naive) independence assumptions between the features.
 \item \textbf{K Nearest Neighbors (KNN - $K=8$)} is a non-parametric method used for classification and regression. In both cases, the input consists of the $k$ closest training examples in the feature space. Since $k$ is even, for tie breaking a random result is selected.
 \item \textbf{Decision Trees (DT - CART, Split Criteria=Entropy)} are a decision support tool that use a tree-like graph or model of decisions and their possible consequences, including chance event outcomes, resource costs, and utility. It is one way to display an algorithm.
 \item \textbf{Random Forests (RF - Split Criteria=Entropy)} are an ensemble learning method for classification, regression and other tasks, that operate by constructing a multitude of decision trees at training time and outputting the class that is the mode of the classes (classification) or mean prediction (regression) of the individual trees. 
\ei

Similar Ghotra et al.~\cite{ghotra2015revisiting}, we used $k=8$ for $k$ Nearest Neighbours. Also for Decision trees and Random Forest we used Entropy as split criteria. Most of these implementations are provided in Scikit-Learn~\cite{pedregosa2011scikit} and available open source.

\subsection{\textbf{Tuning SMOTE using DE (SMOTE2)}}
\label{sect:tuning}

%\newpage
DE  adjusts the parameters of SMOTE given in
Table~\ref{tb:tuned}. Most of these parameters are explained below. 

\begin{table}[!htbp]
    \begin{center}
\scriptsize
\begin{tabular}{|c|c|c|p{3.5cm}|}
        \hline 
        \textbf{Parameters} & \textbf{Defaults} & \textbf{Tuning Range} & \textbf{Description}\\
        \hline
        $k$ & 5 & [1,20] & Number of neighbors in SMOTE \\ 
        \hline
       $m$ & 50\% & {50,100,200,400} & Number of synthetic examples to create. \\ 
        \hline
        $r$ & 2 & [0.1,5] & Power parameter for the Minkowski distance metric.\\

        \hline
\end{tabular}
\end{center}
\caption{List of parameters tuned by this paper}
\label{tb:tuned}
\vspace{-0.4cm}
\end{table}
 
In Table~\ref{tb:tuned}, $k$ controls how many neighbors to choose in minority target class such that synthetic examples can be created between them. It is important to select the number of synthetic examples to create ($m$) and how much undersampling ($m$) of majority class needs to be done. In this case number of oversampling and undersampling are the same. To select neighbors, power ($r$) of Minkowski distance metric is also tunable.

Algorithm 1 is a pseudocode of generic DE. In the following description,
superscript numbers denote lines in the pseudocode. DE evolves a \textit{NewGeneration} of
candidates from a current population.  Each candidate solution in the population is a pair of
(Tunings, Scores). Tunings are selected from Table \ref{tb:tuned} and one of the scores
come from \tion{measure}. Here, the runtimes comes from $\mathit{iter} * np $ evaluations of tuned experiment. The goal of this DE is to either maximize one to these measures like precision, recall, and AUC scores or to minimize the false alarm score. This goal is evaluated on rest data (4 bins). DE is terminated after 3 generations of population which is recommended by Agrawal et al~\cite{agrawal2016wrong}.

The main loop of DE$^{9}$ runs over the \textit{Population}, replacing old items with new Candidates (if new candidate is better).
DE generates \textit{new Candidates} via 
extrapolating$^{23}$ between current solutions in the frontier. Three solutions $a$, $b$ and $c$ are
selected at random. For each tuning parameter i, at some probability \textit{cr}, we
replace the old tuning $x_i$ with $y_i$. For booleans, we use $y_i = x_i\ ^{30}$. For numerics, $y_i = a_i + f \times (b_i - c_i)$ where $f$ is a
parameter controlling crossover. The trim function$^{33}$ limits the new value
to the legal range min..max of that parameter.

\renewcommand{\algorithmicrequire}{\textbf{Input:}}
\renewcommand{\algorithmicensure}{\textbf{Output:}}
\begin{algorithm}
 \scriptsize
    \begin{algorithmic}[1]
    \Require $np=10$, $f=0.7$, $cr=0.3$, $iter=3$, Goal $\in$ Finding maximum/minimum score
    \Ensure $Score, final\_generation$
    \Function{DE}{$np,f,cr,iter, Goal$}
        \State  $Cur\_Gen \leftarrow \emptyset$
        \State $Population \leftarrow $InitializePopulation(np)
        \For{$i = 0$ to $np-1$}
            \State $Cur\_Gen$.add($Population$[i],score)
        \EndFor
        \For{$i = 0$ to $iter$}
            \State $NewGeneration \leftarrow \emptyset$
            \For{$j = 0$ to $np-1$}
                \State $S_i \leftarrow $Extrapolate($Population$[j],Population,cr,f,np)
                \If{score($S_i$) $\geq$ $Cur\_Gen$[j][1]}
                    \State $NewGeneration$.add($S_i$,score($S_i$))
                \Else
                    \State $NewGeneration$.add($Cur\_Gen$[j])
                \EndIf
            
            \EndFor
            \State  $Cur\_Gen \leftarrow NewGeneration$
        \EndFor
        \State $Score \leftarrow$ GetBestSolution($Cur\_Gen$)
        \State  $final\_generation \leftarrow Cur\_Gen$
        \State \textbf{return} $Score, final\_generation$
    \EndFunction

    \Function{Extrapolate}{$old, pop, cr, f,np$}
        \State $a,b,c \leftarrow threeOthers$(pop, old)
        \State $newf \leftarrow \emptyset$
        \For{$i = 0$ to $np-1$}
            \If{$cr \leq$ random()}
                \State $newf$.add($old[i]$)
            \Else
                \If{typeof($old$[i])$ ==$ bool then}
                    \State $newf$.add(not $old[i]$)
                \Else 
                    \State $newf$.add(trim(i,($a$[i]+$f\ast$($b$[i] $-$ $c$[i]))))
                \EndIf
            \EndIf
        \EndFor
        \State \textbf{return} $newf$ 
    \EndFunction
    \caption{Pseudocode for DE with a constant number of iterations}
    \end{algorithmic}
\end{algorithm}

The loop invariant of DE is that, after the i-th iteration$^7$, the \textit{Population}
contains examples that are better than at least one other candidate.
As the looping progresses, the \textit{Population} is full of increasingly more valuable solutions
which, in turn, also improve the candidates, which are extrapolated from the Population.
Hence, Vesterstrom et al.~\cite{vesterstrom2004comparative} found DE to be
competitive with particle swarm optimization and other GAs.

\subsection{\textbf{Evaluation Measures}}
\label{sect:measure}

Since, this is a binary classification problem, we represent the predictions using a confusion matrix where a `positive' output is the defective class under study and a `negative' output is the non defective class. The confusion matrix is shown in Figure \ref{fig:cmatrix}.

\begin{figure}[!htpb]
\begin{center}
\begin{tabular} {@{}cc|c|c|l@{}}
\cline{3-4}
& & \multicolumn{2}{ c| }{Actual} \\ \cline{3-4}
& & $p$ & $n$  \\ \cline{1-4}
\multicolumn{1}{ |c  }{\multirow{2}{*}{Predicted} } &
\multicolumn{1}{ |c|| }{$p'$} & $TP$ & $FP$ & \\ \cline{2-4}
\multicolumn{1}{ |c  }{}                        &
\multicolumn{1}{ |c|| }{$n'$} & $FN$& $TN$  &  \\ \cline{2-4}
\cline{1-4}
\end{tabular}
\caption{Confusion Matrix}
\label{fig:cmatrix}
\end{center}
\vspace{-0.8cm}

\end{figure}

%\begin{figure}[!htpb]
%    \centering
%    \includegraphics[scale=0.35]{cmatrix.png}
%    \caption{Confusion Matrix}%

%    \label{fig:cmatrix}
%\end{figure}
\noindent
We define the measures as
\begin{itemize}
\item \textbf{Area Under Curve} is the area covered by an ROC curve~\cite{swets1988measuring, duda2012pattern} in which the X-axis represents:
\[\%FP = \dfrac{FP}{FP + TN}\]
and the Y-axis represents:
\[\%TP = \dfrac{TP}{TP + FN}\]
\item \textbf{Recall}  is the fraction of relevant instances that are retrieved.
\[Recall(pd) = \dfrac{TP}{TP + FN}\]
\item \textbf{Precision} is the fraction of retrieved instances that are relevant.
\[Precision(prec) = \dfrac{TP}{TP + FP}\]
\item \textbf{False Alarm} is the ratio of false positive to predicted negative total.
\[False\ alarm(pf) = \dfrac{FP}{FP + TN}\]
\end{itemize}

\begin{figure*}[!t]
    \centering
  \begin{minipage}{.49\textwidth}
        \captionsetup{labelsep=space,justification=centering}
        \includegraphics[width=\linewidth]{./fig/AUC_tuned2.png}
  \caption{Area Under Curve}
  \label{AUC}
  \end{minipage}
  \begin{minipage}{.49\textwidth}
        \captionsetup{justification=centering}
        \includegraphics[width=\linewidth]{./fig/Recall_tuned2.png}
  \caption{Recall Scores}
  \label{Recall}
    \end{minipage}%

\end{figure*}

\begin{figure*}[!htbp]
    \centering
      \begin{minipage}{.49\textwidth}
        \captionsetup{justification=centering}
        \includegraphics[width=\linewidth]{./fig/Precision_tuned2.png}
  \caption{Precision Scores}
  \label{Precision}
    \end{minipage}%    
  \begin{minipage}{.49\textwidth}
        \captionsetup{justification=centering}
        \includegraphics[width=\linewidth]{./fig/False_alarm_tuned2.png}
  \caption{False Alarm Scores}
  \label{false}
    \end{minipage}%
\end{figure*}

\section{Results}
\label{sect:results}
Figures ~\ref{AUC},~\ref{Recall},~\ref{Precision} and~\ref{false} represents all the 4 evaluation criteria against the 6 learners. X-axis represents all 7 datasets in the decreasing percentage of defective classes from left to right. The corresponding percentage is written beside each dataset. Y-axis represents absolute delta improvement between 2 results. Dashed lines represent absolute delta improvement between SMOTE1 - No SMOTE results. Solid lines represent absolute delta improvement between SMOTE2 - No SMOTE results. For figures ~\ref{AUC},~\ref{Recall},~\ref{Precision}, more positive the better and if negative then worse. But for figure ~\ref{false}, it is the other way around. Negative represents better and more positive shows worse results.

\subsection{\textbf{RQ1: Is standard ``off-the-shelf'' SMOTE1 with their default tunings recommended for defect prediction?}}

From all the 4 figures mentioned above, we are seeing about 25\% improvement in AUC for 3 learners which are KNN, LR and SVM (Figure~\ref{AUC}). For other learners the improvements are either 0 or becomes negative by 5\%. It is recommended to use SMOTE1 for these 3 learners. For recall, doing SMOTE1 is highly recommended for imbalanced datasets. All 6 learners kept performing better and better as defective class percentage went minor and minor (Figure~\ref{Recall}). This is what was expected for applying SMOTE.

In terms of precision (Figure~\ref{Precision}), we are seeing a decreasing performance for most datasets against most learners. This concludes we shouldn't be using SMOTE1 if the goal is to have better precision. If we used SMOTE, then it comes with an expense of increasing false alarm (Figure~\ref{false}).

\begin{lesson}
    For defect data, SMOTE1 has little effect on 
 precision, modest improvements for AUC(pf,recall) and largest improvements for recall.
\end{lesson}

\subsection{\textbf{RQ2: Is tuned SMOTE better than the standard untuned SMOTE?}}

From all the 4 figures mentioned above, we are seeing minimum of 50\% improvement than SMOTE1 for AUC for all 6 learners (Figure~\ref{AUC}). It is highly recommended to use SMOTE2 if the tuning goal is to improve higher AUC. For recall, doing SMOTE2 has good improvements than SMOTE1. And all 6 learners kept performing better and better as defective class percentage went minor and minor (Figure~\ref{Recall}). This is what was expected for applying SMOTE.

In terms of precision (Figure~\ref{Precision}), we are seeing a decreasing performance for most datasets against most learners which is similar to what was observed with SMOTE1 results. This concludes we shouldn't be using SMOTE2 if the goal is to have better precision. And if we used SMOTE2, then it comes with some not large false alarm increase (Figure~\ref{false}).

\begin{lesson}
    For defect data, SMOTE2  
 offers   large  improvements over SMOTE1 for recall
 and dramatic improvements for AUC (pf,recall).
\end{lesson}

\subsection{\textbf{RQ3: Do different data sets
      need different configurations with SMOTE2?}}

abc

\begin{lesson}
    DE finds different ``best'' parameter settings for SMOTE for different data sets. Hence reusing tunings  suggested  by  any other  previous study  for any dataset is \underline{{\em not}} recommended. Instead,  it is better to
      use  automatic  tuning  methods  to find the best tuning parameters for the current data set.
\end{lesson}

   \textbf{RQ4}: \textbf{Is tuning extremely slow?} 
 
 \begin{lesson}Tuning with DE makes training three to five times slower, but the improvements which we get for AUC and recall is quite advantageous.
 \end{lesson}
 
 

\subsection{\textbf{RQ4: Is tuning extremely slow?}}

abc

\begin{lesson}
    Tuning with DE makes training three to five times slower, but the improvements which we get for AUC and recall is quite advantageous.
\end{lesson}



\section{Threats to Validity}
\label{sect:validity}

As with any empirical study, biases can affect the final
results. Therefore, any conclusions made from this work must be considered with the following issues in mind:

\textbf{\textit{Sampling bias}} threatens any classification experiment; i.e., what matters there may not be true here. For example, the datasets used here comes from the PROMISE~\cite{promiserepo} repository and were supplied by one individual. But these datasets have used in various case studies by various researchers~\cite{he2012investigation,peters2013better,peters2013balancing,turhan2013empirical} making our results more conclusive.
Though we have used 9 open-source datasets for Software Defect prediction (Table~\ref{tb:dataset}) which are mostly from Apache, and some are proprietary projects, there could be other datasets for which our results could be wrong.

\textbf{\textit{Learner bias}}: For building the defect predictors in this
study, we selected each learner with default parameters like k=8 in $k$-NN, entropy as split criteria in RF, and DT. The above predefined parameters have been used in the conclusions made by other studies~\cite{ghotra2015revisiting,tantithamthavorn2016automated}. But classification is a large and active field and any single study can only use a small subset of the known classification algorithms.

\textbf{\textit{Evaluation bias}}: Most highly cited papers~\cite{ghotra2015revisiting,tantithamthavorn2016automated} have only concluded using $AUC\ (pf, recall)$ whereas we went ahead and are reporting 3 other measures. There are other measures used in software engineering which includes Accuracy, Fscore. But we still performed SMOTE2 on Fscore and Accuracy and did not find any improvement (Figure~\ref{fig:threats}). Anyway, accuracy is not the best measure to report for defect prediction as we do not want any true negative numbers included. Fscore is the harmonic mean of precision and recall and if you recall our SMOTE2 results, we see  increment in recall as well as  decrement in precision making Fscore negligible.

\begin{figure}[!htbp]
\begin{minipage}{\linewidth}
\centering
        \includegraphics[width=.95\linewidth]{./fig/acc_f_tuned.png}
    \end{minipage}%
    \caption{SMOTE2 improvement over SMOTE1. Legends represent the classifiers mentioned in \tion{classes}}
    \label{fig:threats}
\vspace{-0.5cm}
\end{figure}

\textbf{\textit{Order bias}}: With each dataset how data samples are distributed in training and testing set is completely random. Though there could be times when all good samples are binned into training and testing set. To mitigate this order bias, we run
the experiment 25 times by randomly changing the order of the data samples each time.

\section{Conclusion}
\label{sect:conclusion}

Based on the above, we offer a specific and general conclusion. Most specifically, we recommend.
\bi
 %\item Previous studies~\cite{ghotra2015revisiting,tantithamthavorn2016automated} have mostly reported on only 1 evaluation goal, but we reported on many.
 \item Any study~\cite{ghotra2015revisiting} that has reported AUC without studying the effects of class imbalance needs to be analyzed again by applying SMOTE. We say this since SMOTE2 showed big improvements in AUC.
% \item Whenever we want to maximise a certain evaluation goal, specifically AUC and recall, then it is recommended to tune the parameters of SMOTE1.
 \item Unlike the advise of Chawla et al.~\cite{chawla2002smote} of using $k=5$ for SMOTE, we should have automatic methods like DE to tune the parameters of SMOTE and find the best parameter settings.
 \item Do not use someone else's pre-tuned SMOTE, since, as shown here, the best tunings vary from dataset to dataset.
\ei
Based on our runtime results, we can say that list last
recommendation (to tune before applying analytics) is not  a particular
onerous demand. The additional cost of tuning
is relatively minor while the benefits of the SMTOE2 tunings
are very large (see the   AUC and recall results shown above).

% In the future, we plan to work on other similar issues like:
% \bi
%  \item Tuning SMOTE and learners at the same time and comparing against the results of No SMOTE, SMOTE1, and SMOTE2.
%  \item Predicting the quantities of defects which is regression based model rather having a binary based classification model with tuning.
% \ei


\balance

\bibliographystyle{abbrv}

%\bibliographystyle{ACM-Reference-Format}
\medskip
\bibliography{main}

\end{document}